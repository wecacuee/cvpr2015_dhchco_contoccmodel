\documentclass[final]{beamer}
% The following are possible:
%\documentclass[landscape,a0b,final,a4resizeable]{egsposter2013}
%\documentclass[portrait,a0b,final,a4resizeable]{egsposter2013}
%\documentclass[portrait,a0b,final]{egsposter2013}

% Set font
% If this will not compile, try commenting this out temporarily
\usepackage{amsmath}
\usepackage{fontspec}
\setmainfont{Arial}

% The font sizes seem to be inconsistent between TeX and MS Powerpoint.
% These are corrected below for the TeX template.

% Graphics path
\usepackage{graphicx}
\graphicspath{{figs/}}

% Load the EGS style
%\usepackage{egsposter}
\usepackage[scale=1.24]{beamerposter}

% Some packages used for the template
\usepackage{multicol}
\usepackage{graphicx}

\DeclareMathOperator*{\argmin}{arg\,min}
\DeclareMathOperator*{\argmax}{arg\,max}
\newcommand{\vect}[1]{\mathbf{#1}}
\newcommand{\hvect}[1]{\bar{\vect{#1}}}
\newcommand{\uvect}[1]{\hat{\vect{#1}}}
\newcommand{\field}[1]{\mathbb{#1}}
\newcommand{\Real}[0]{\field{R}}

\newcommand{\map}{\vect{x}}
\newcommand{\meas}{z}
\newcommand{\measurements}{\vect{\meas}}
\newcommand{\pose}{g}
\newcommand{\poses}{\vect{\pose}}
\newcommand{\unaryminus}{\scalebox{0.5}[0.5]{\( - \)}}
\newcommand{\prevtime}{1:t\unaryminus1}
%\newcommand{\pastobs}{\sigma_{\prevtime}}
\newcommand{\pastobs}{\measurements_{1:t\unaryminus1}, \poses_{1:t\unaryminus1}}
\newcommand{\remaining}{\measurements_{\prevtime}, \poses_{\prevtime}}
\newcommand{\bpmsg}[4]{\mu^{#4}_{#1\rightarrow#2}(#3)}

\newcommand{\msg}[3]{\mu_{#1#2}(#3)}
\newcommand{\assign}{\leftarrow}
\newcommand{\Sx}{L_i}

\newcommand{\actz}{\bar{z}_f(\map_f, g_f)}

% Poster title
\newcommand{\postertitle}{Modern MAP inference methods for accurate and fast
occupancy grid mapping on higher order factor graphs}

% Poster authors
\newcommand{\authors}{
Vikas Dhiman\textsuperscript{1},
Abhijit Kundu\textsuperscript{2},
Frank Dallaert\textsuperscript{2},
Jason J. Corso\textsuperscript{1}}

% Authors' departments
\newcommand{\departments}{
\textsuperscript{1}Electrical Engineering and Computer Science, Univeristy of Michigan, MI, USA,
\textsuperscript{2}College of Computing, Georgia Tech, GA, USA}


\begin{document}

%\begin{poster}

  % Header
  \maketitle

  \begin{multicols}{3}

    % Add a section heading
    % -----------------------
    \sectionheading{Introduction}
    % -----------------------

    % Note the usage of 0.5in hspace between the minipages. This is required (see text below)
    \begin{minipage}[c]{0.4\columnwidth}
      Using the inverse sensor model has been popular in occupancy grid mapping
      which assumes independence cell assumption, weakening the model. In our
      work, we make use of modern MAP inference methods along with the forward
      sensor model. Our implementation and experimental results demonstrate
      that these modern inference methods deliver more accurate maps more
      efficiently than previously used methods.
    \end{minipage}
    \hspace{0.5in}
    \begin{minipage}[r]{\dimexpr0.6\columnwidth-0.5in\relax}
      \begin{center}
      \includegraphics[width=\columnwidth]{../../figures/factorgraph/factorgraph3.pdf}
      \captionof{figure}{Representation of occupancy grid mapping as factor graph}
      \label{fig:factor-graph}
      \end{center}
    \end{minipage}

    % -----------------------
    \sectionheading{Contributions}
    % -----------------------

    % The EGS template defined packedEnumerate and packedItemize for convenience
    \begin{packedEnumerate}
      \item We introduce factor graph and apply modern fast inference
        algorithms, such as loopy belief propagation
        \cite{kschischang2001factor} and dual decomposition
        \cite{sontag2011introduction} to the problem of occupancy grid mapping.

      \item We introduce a class of higher order factors for the factor graph
        approach for which it is possible to reduce the time complexity of algorithm to polynomial
        time from exponential in neighborhood size.

    \end{packedEnumerate}

    % -----------------------
    \sectionheading{Mapping with an inverse sensor model}
    % -----------------------

    Mapping with inverse sensor model assume that each grid cell is independent of all other map cells: 
    \begin{align}
      p(\map|\measurements, \poses) &= \prod_{1 \le i \le N} p(x_i|\measurements, \poses)
      \enspace. \\
      p(x_i|\measurements, \poses) &= \frac{1}{Z'p^{t-1}(x_i)}\prod_{1\le f \le t} p(x_i|\meas_f, \pose_f)\enspace.
    \end{align}

    % -----------------------
    \sectionheading{Mapping with a forward sensor model}
    % -----------------------

    In forward sensor model we instead assume that each sensor measurement is independent of each other. 
    \begin{align}
     p(\map | \measurements, \poses) &= \frac{1}{Z} \prod_{1 \le f \le t}
     p(z_f|\map_f, g_f)\enspace.
     \label{eq:modernmap}
    \end{align}

    % Add columnbreak
    \vfill\columnbreak

    % -----------------------
    \sectionheading{Efficient Belief propagation}
    % -----------------------
    Belief propagation works by message passing over factor graph edges

\begin{align}
  \bpmsg{f}{i}{l_i}{r+1} &= \sum_{\map_f \in \Omega_f: x_i = l_i}P_f(\map_f)\prod_{j \in n(f) \setminus i}\bpmsg{j}{f}{x_j}{r}
  \label{eq:factor2node}
  \\
  \bpmsg{i}{f}{l_i}{r+1} &= \prod_{h \in n(i) \setminus f}\bpmsg{h}{i}{l_i}{r}
  \enspace,
  \label{eq:node2factor}
\end{align}
    For efficiency we suggest a class of factors called
    \textit{Pattern based factors} 
    \begin{align}
      p(z_f|\map_f, g_f) = \begin{cases}
        \psi_{m} & \text{ if $\map_f \sim \vect{R}_m$} \,\,\, \forall 1 \le m \le M\\
        \psi_{\text{min}} & \text{ otherwise}
      \end{cases}
      \enspace,
      \label{eq:patternfunction}
    \end{align}
    \begin{align}
      p(\vect{x_f} \sim \vect{R}_m|x_i = l_i)
      = \begin{cases}
    0 & \text{if $i \in n^m_0(f)$ and $l_i \ne r^m_i$}\\
      \prod\limits_{j \in n^m_0(f) \setminus i}\bpmsg{j}{f}{r^m_j}{r} & \text{otherwise}
      \end{cases}.
    \end{align}
    \begin{align}
      \bpmsg{f}{i}{l_i}{r+1} =
      \sum_{m \le M} \psi_m p(\vect{x_f} \sim \vect{R}_m|x_i = l_i)
      + \psi_{\text{min}} p_{\text{otherwise}}
      \label{eq:efficientbp}
      \enspace.
    \end{align}
    where $p_{\text{otherwise}} = 1 - \sum_{m \le M}p(\vect{x_f} \sim \vect{R}_m|x_i = l_i)$.  The message update can be computed in $O(M|n(f)|)$ using
    \eqref{eq:efficientbp}, instead of $O(L_i^{|n(f)|})$ in the general case.

    % -----------------------
    \sectionheading{Results and Discussion}
    % -----------------------


    \begin{minipage}[c]{1.0\columnwidth}
      \begin{center}
       \includegraphics[height=0.20\columnwidth, angle=90]{../../figures/zoomAndShow/zoomAndShowTwoAssumptionAlg.pdf}%
       \includegraphics[height=0.20\columnwidth, angle=90]{../../figures/zoomAndShow/zoomAndShowSICKDDMCMC.pdf}%
       \includegraphics[height=0.20\columnwidth, angle=90]{../../figures/zoomAndShow/zoomAndShowSICKDDMCMC.pdf}%
       \includegraphics[height=0.20\columnwidth, angle=90]{../../figures/zoomAndShow/zoomAndShowBP.pdf}%
       \includegraphics[height=0.20\columnwidth, angle=90]{../../figures/zoomAndShow/zoomAndShowDD.pdf}%
       \captionof{figure}{Qualitative results with different algorithms (from
         left to right): 1) Ground truth with the trajectory of the robot 2)
         Inverse sensor model 3) Metropolis Hastings without heat map 4)
         Metropolis Hastings with heat map 5) Belief Propagation (BP) 6) Dual
       decomposition (DD)}
      \end{center}
    \end{minipage}
    \providecommand{\mysubfloatwidth}{0.33\columnwidth}
    \begin{minipage}[c]{1.0\columnwidth}
      \begin{center}
        \includegraphics[width=\mysubfloatwidth]{../../../Data/cave_player/plot-time-energy.pdf}%
        \includegraphics[width=\mysubfloatwidth]{../../../Data/hospital_section_player/plot-time-energy.pdf}%
        \includegraphics[width=\mysubfloatwidth]{../../../Data/albertb.sm/plot-time-energy.pdf}%
        \captionof{figure}{Comparison of convergence rate of different
        algorithms on occupancy grid mapping on three datasets. While sampling methods like
      Metropolis hastings converge quickly they stay far from optimum energy. }
      \end{center}
    \end{minipage}

    % Add columnbreak
    \vfill\columnbreak

    % -----------------------
    \sectionheading{Conclusion}
    % -----------------------

    \begin{packedEnumerate}
        \item Using forward sensor models is a better then using inverse sensor
          model for occupancy grid mapping.
        \item Belief propagation and dual decomposition are faster algorithms
          as compared to Metropolis Hastings and converge to lower minima.
        \item Pattern based factors can be formulated for efficient inference.
    \end{packedEnumerate}
    
    % -----------------------
    \sectionheading{Future work}
    % -----------------------
    \begin{packedEnumerate}
      \item Extend the forward sensor model approach for localization and mapping.
      \item A recent comparative study \cite{kappes2013comparative} provides
        several candidate methods that can be used with higher order factors.
    \end{packedEnumerate}


    % -----------------------
    \sectionheading{Acknowledgements}
    % -----------------------

    Vikas Dhiman and Jason J. Corso were partially supported by FHWA
    DTFH61-07-H-00023 and NSF CAREER IIS-0845282 and ARO YIP W911NF-11-1-0090.
    Frank Dellaert and Abhijit Kundu
    were partially supported by an ARO MURI W911NF-11-1-0046.
    We thank Stan
    Birchfield and Brian Peasley for discussions and early efforts in this
    work.

    % -----------------------
    \sectionheading{References}
    % -----------------------

    %\nocite{*}
    {\onehalfspacing\scriptsize
      \bibliographystyle{plain}
      \bibliography{../../icra2014/modern_map}}

    % Add the sponsors
    \addegssponsors

  \end{multicols}

%\end{poster}

\end{document}
