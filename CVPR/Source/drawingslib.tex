
\providecommand{\dscale}{0.5}
% Image frame
\newcommand{\drawframe}{
  %\path[use as bounding box,draw] (0.5,-2) rectangle (10,2);

\coordinate (A\x) at (\x*1.2,0);
\coordinate (a\x) at (\x*1.2+0.3,0+0.5);

\draw [thick,black] (A\x) -- +(0,1) -- +(1,1.5) -- +(1,.5) -- cycle;


\draw [very thick,blue!40,fill=blue!20] (a\x) -- +($\dscale*(0,1)$) node (b\x) {} -- +($\dscale*(1,1.5)$) node (c\x) {} -- +($\dscale*(1,.5)$) node (d\x) {} -- cycle;

}

\newcommand{\road}{
    % Road width
    \newcommand{\rw}{0.15};
    % Road slant
    \newcommand{\rslant}{0.15};
    % Upper coordinate of the road
    \pgfmathsetmacro{\rup}{\rw+\rslant};
    % Where does the road start on the left
    \newcommand{\rxmin}{6};
    % length of the  road
    \newcommand{\rxlen}{5};
    % right end of the road
    \pgfmathsetmacro{\rxmax}{\rxmin+\rxlen};

    % Draw the road
    \filldraw[fill=black!20, draw=black!40] 
    (3d cs:\rxmin, 0,  0) -- (3d cs:\rxmax, 0, \rslant)
    -- (3d cs:\rxmax, 0,  \rup) -- (3d cs:\rxmin, 0, \rw) -- cycle;
    % Middle of the road coordinates
    \pgfmathsetmacro{\rmid}{\rw/2};
    \pgfmathsetmacro{\rmidr}{\rslant + (\rw)/2};
    % Dashed line in the middle of the road
    \draw[white,very thick,dashed] (3d cs:\rxmin, 0, \rmid ) -- (3d cs:\rxmax, 0, \rmidr);
}

\newcommand{\acuboid}{
  \providecommand{\acxmin}{-2}
  \providecommand{\acxlen}{1}
  \providecommand{\acymin}{-2}
  \providecommand{\acylen}{1}
  \providecommand{\aczmin}{-2}
  \providecommand{\aczlen}{1}
  \pgfmathsetmacro{\acymax}{\acymin+\acylen}
  \pgfmathsetmacro{\acxmax}{\acxmin+\acxlen}
  \pgfmathsetmacro{\aczmax}{\aczmin+\aczlen}
  \coordinate (a) at (3d cs:\acxmax,\acymax,\aczmin);
  \coordinate (b) at (3d cs:\acxmax,\acymax,\aczmax);
  \coordinate (c) at (3d cs:\acxmax,\acymin,\aczmax);
  \coordinate (d) at (3d cs:\acxmax,\acymin,\aczmin);
  \coordinate (h) at (3d cs:\acxmin,\acymin,\aczmin);
  \coordinate (e) at (3d cs:\acxmin,\acymax,\aczmin);
  \coordinate (f) at (3d cs:\acxmin,\acymax,\aczmax);

  \draw [facestyle] (a) -- (b) -- (c) -- (d) -- cycle;
  \draw [facestyle] (a) -- (b) -- (f) -- (e) -- cycle;
  \draw [facestyle] (a) -- (d) -- (h) -- (e) -- cycle;
  % For debugging turn this on to see corners labeled
  %\path (a) node {a} (b) node {b} (c) node {c} (d) node {d}  (h) node {h} (e) node {e} (f) node {f};
}

\newcommand{\car}{
\providecommand{\cxmin}{0}
\providecommand{\cxmax}{1}
\providecommand{\czmin}{0}
\providecommand{\czmax}{1}
\providecommand{\cyzmin}{0}
\providecommand{\cyzmax}{0}
\providecommand{\cylen}{1}
    \path (3d cs:\cxmin, 0, \czmin) coordinate (a)
    (3d cs:\cxmax, 0, \czmax) coordinate (b)
    (3d cs:\cxmax, \cylen, \czmax) coordinate (c)
    (3d cs:\cxmin, \cylen, \czmin) coordinate (d)
    (3d cs:\cxmax, \cylen, \cyzmax) coordinate (e)
    (3d cs:\cxmin, \cylen, \cyzmin) coordinate (f)
    (3d cs:\cxmin, 0, \cyzmin) coordinate (g)
    ;

  % front face
  \filldraw[facestyle] (a) -- (b) -- (c) --  (d) -- cycle;
  % upper face
  \filldraw[facestyle] (c) -- (e) -- (f) --  (d) -- cycle;
  % left (wrt viewer) face
  \filldraw[facestyle] (a) -- (d) -- (f) --  (g) -- cycle;
  \node (lB) at ($(e) + (0,.2)$) {$\dimsn{i}$};
}

\newcommand{\caronroad}{
  % Draw the road and the car
  \begin{scope}[3d/perspective eye={0,5,-1},facestyle/.style={fill=blue!20, draw=blue!40}]    
    %\path[use as bounding box,clip] (5.5,-1) rectangle (8,3);
    \road{}

    % Displacement of car w.r.t road
    \newcommand{\cxdisp}{1};
    % left most coordinate of the car
    \pgfmathsetmacro{\cxmin}{\cxdisp+\rxmin};
    % length of car
    \newcommand{\cxlen}{1.5};
    % height of car
    \newcommand{\cylen}{.8};
    % width of car
    \newcommand{\czlen}{.4*\rw};
    % right most x-coordinate of car
    \pgfmathsetmacro{\cxmax}{\cxmin+\cxlen};
    % z-min coordinate of car
    \pgfmathsetmacro{\czmin}{\rslant/\rxlen*\cxdisp+0.02};
    % z-max
    \pgfmathsetmacro{\czmax}{\rslant/\rxlen*(\cxmax-\rxmin)+0.02};
    % z-max on the farther side
    \pgfmathsetmacro{\cyzmax}{\czmax+\czlen};
    % z-min on the farther side
    \pgfmathsetmacro{\cyzmin}{\czmin+\czlen};

    \car{}
  \end{scope}
}

% Camera
\newcommand{\camera}{
\path (5,1) coordinate (c0) 
 +(0.25,0.15) coordinate (c2);
  \draw (c0) rectangle (c2);
  \draw (c2) -- ++(0.075, 0.075) -- ++(0, -0.3) -- ++(-0.075,0.075) ;
}
