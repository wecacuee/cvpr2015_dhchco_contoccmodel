
% \begin{tight_itemize}
%   \item Introduce the problem
%   \item Describe our approach
%   \item Contrast with Milan's work
%   \item Contrast with Andreas Geiger's work
%   \item Highlight the contributions
% \end{tight_itemize}
% \vspace{10cm}

        To accomplish fully or partial autonomous driving we need 3D
        localization of traffic participants in the scene. 
        Since laser scanners and stable wide--baseline stereo setups are
        expensive, we focus on solving the problem using monocular video, GPS
        and maps as our input. Given the video we extract detection bounding
        boxes, ground plane and point tracks.
        We formulate the problem of 3D localization in a probabilistic graphical
        model specifically a factor graph model. We use additional heuristic
        constraints like collision constraint, that leads to a global consistent solution.
        Using these multiple sources of information that provide, possibly conflicting evidence about
        the location of traffic participant, a consistent meaningful solution must be estimated. 
        We formulate the problem in factor graph
        formalism and use off-the-shelf methods for inference.

        Our main contribution is a novel continous occlusion aware 3D object
        modeling. This modeling is more principled than the continuous occlusion models proposed by Milan et al~\cite{Milan_etal_2014}, as our model is in 3D and derives it reasoning from fundamentals of optics. 
         In other words, we provide a optics based
        perspective to occlusion modeling. Also we propose a unique point
        tracks energy that is based on reprojection error and occlusion based
        association probability. Our experiments show that association
        probability provides more accurate point to object association than
        using simple bounding box based methods.

