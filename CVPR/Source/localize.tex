%%%%%%%%%%%%%%%%%%%%%%%%%%%%%%%%%%%%%%%%%%%%%%%%%%%%
\section{Continuous Energies for Object Localization}
%%%%%%%%%%%%%%%%%%%%%%%%%%%%%%%%%%%%%%%%%%%%%%%%%%%%
We explain energies in this section.

%%%%%%%%%%%%%%%%%%%%%%%%%%%%%%%%%%%%%%%%%%%%%%%%%%%%
\subsection{Point tracks energy with occlusion}
\label{sec:totalContPtTracksEnergy}
We model continuous point tracks energy with explicit occlusion reasoning as
the expected re-projection error over the association probability,

\begin{multline}
  \Energy{traci}(\{ \relp{i}{t} \}_i, \{ \relp{i}{t-1} \}_i, \{\dimsn{i}\}_i ) = \\
    \sum_{i=1}^{N} 
    %\sum_{t = s_i}^{e_i}
    \sum_{j = 1}^{M}
    \int_1^\infty \assocP\Ereproj(\lambda) d\lambda
\end{multline}
where $\assocP$ is the association probability of
$j$\textsuperscript{th} point with $i$\textsuperscript{th} TP at depth $\lambda$
and $\Ereproj(\lambda)$ is the re-projection error given by
%
\begin{align}
  \assocP &= \Prefl\Ptrans\\
  \Ereproj(\lambda) &= \left\|\trackpj{t+1} - \projectionOft{\invProjectionOf{\trackpj{t}, \lambda}}\right\|^2 .
  \label{eq:reprojerror}
\end{align}

The  $\projectionOf{.}$ and $\invProjectionOf{.}$ denote the projection and
inverse projection functions that project 3D point to camera image and vice
versa. Note that inverse projection $\invProjectionOf{.}$ depend on both the
point $\trackp{t}$ and the unknown depth $\lambda$. Also note that the inverse projection is dependent on TP pose at time $t$ while the projection depends on pose at time $t+1$ which can be different.

\begin{comment}
  \subsubsection{Occupancy function}
  Assuming occupancy to be a
  probability distribution over 3D space. For each TP the
  occupancy is modeled as a logistic function 
  \begin{align}
     \occf = L(\mathbf{x}; \pos{i}{t}, \Sigma_i)
  \end{align}
  where $L(.)$ is the logistic function defined by
  \begin{align}
    L(\mathbf{x}; \pos{i}{t-1}, \Sigma_i) = \frac{1}{
      1 + e^{-k(1 - d(\mathbf{x},\pos{i}{t-1}))}
      }
  \end{align}
  where $d(\mathbf{x},\pos{i}{t-1}) =
  (\mathbf{x}-\pos{i}{t-1})^\top\Sigma_i(\mathbf{x}-\pos{i}{t-1})$ and $k =
  10\ln{49}$. $k$ is chosen such that $L(.) = 0.98$ when $d(.) =
  0.9$
  % Once we have our distribution over $\lambda$, $\lambdadist$ we can compute the
  % reprojection of $\trackpj{t-1}$ over image in time $t$ as a function of
  % $\lambda$. Let
\end{comment}


\begin{comment}
  \subsubsection{Approximations}
  Reflection probability of $i$th TP is easy to compute
  analytically 
  \begin{multline}
    \Prefl =
    (\max \{ 0, \nabla \occf^\top \ray \})^2 \\
    = (\max \{ 0, \nabla \occfxi^\top\ray \})^2
    \label{eq:analytic-prefl}
  \end{multline}
  where 
  \begin{multline}
    \nabla \occfxi^\top\ray \\=
    \nabla k\dishort \sech^2\left(\frac{k}{2}\dishort\right)
  \end{multline}
  where $\dishort = 1-d(\mathbf{x}, \pos{i}{t-1})$ is a signed distance measure
  from the contour of the ellipsoid where $d(\mathbf{x}, \pos{i}{t-1})$ is 1.

  However, the transmission probability needs to be approximated.  
  %
  % \subsubsection{Computing $\Prefl$ and $\Ptrans$}
  % Focusing on  $\nabla \occfxi$ 
  % 
  % \begin{multline}
  %   \nabla \occfxi =\\
  %   \frac{-k\nabla d(\mathbf{x}, \pos{i}{t-1})e^{-k(1-d(\mathbf{x}, \pos{i}{t-1}))}}{
  %     (1 + e^{-k(1-d(\mathbf{x}, \pos{i}{t-1}))})^2
  %   } \\ 
  %   = -k\nabla d(\mathbf{x}, \pos{i}{t-1})\sech^2\left(\frac{k}{2}(1-d(\mathbf{x},
  %   \pos{i}{t-1}))\right)
  %   \\
  %   = \nabla k\dishort \sech^2\left(\frac{k}{2}\dishort\right)
  % \end{multline}
  % where $\dishort = 1-d(\mathbf{x}, \pos{i}{t-1})$ is a signed distance measure
  % from the contour of the ellipsoid where $d(\mathbf{x}, \pos{i}{t-1})$ is 1.
  % Focusing on $\nabla d(.)$
  % 
  % \begin{align}
  %   \nabla d(\mathbf{x}, \pos{i}{t-1}) = 2\Sigma_i(\mathbf{x} - \pos{i}{t-1})
  % \end{align}
  % Back to \eqref{eq:analytic-prefl}
  % 
  % \begin{multline}
  %   (\max \{ 0, \nabla \occf^\top \ray \})^2 = \\
  %   \sech^4\left(\frac{k}{2}\dishort\right) 
  %   (\max \{ 0 , \nabla k\dishort^\top\ray\})^2
  % \end{multline}
  % 
  % The probability is simply $\sech^2(.)$ scaled by gradient of ellipsoid
  % $\nabla k\dishort$ projected in the ray direction $\ray$.
  % 
  % \begin{align}
  %   \Ptrans = 
  %   e^{\int_{1}^{\lambda} \ln{(1 - f_{occ}(\lambda \ray))}{d\lambda}}
  % \end{align}
  % \begin{multline}
  % \int_{1}^{\lambda} \ln{(1 - f_{occ}(\lambda \ray))}{d\lambda}
  % =  \\
  % \int_{1}^{\lambda} \ln{\left(1 - \sum_i \occfi\right)}{d\lambda}
  % \end{multline}
  % 
  % The above integral is very difficult to approximate or compute analytically.
  So based on intuition, we approximate the $\Ptrans$ by following function
  \begin{align}
  \label{eq:evalCumulativePtrans}
    \Ptrans &= \prod_i L_u(\mathbf{u}, \mu^i_u,\Sigma^i_u)L_{\lambda}(\lambda; \mu^i_d)\\
    L_u(\mathbf{u}, \mu^i_u,\Sigma^i_u) &= \frac{1}{
      1 + e^{-k_u(1 - (\mathbf{u} - \mu^i_u)^\top\Sigma^i_u(\mathbf{u} -
      \mu^i_u))}
    }
    \\
    L_{\lambda}(\mathbf{u}, \lambda; \mu^i_d) &= \frac{1}{
    1 + e^{-k_d(\lambda - \mu^i_d(\mathbf{u}))}
  }
  \end{align}
  where 
  \begin{align}
    \mu_u^i &= \projectionOf{\pos{i}{t-1}} \label{eq:muiudef}\\
    \Sigma_u^i &= \projectionOf{\Sigma_i} \label{eq:sigmauidef}\\
    \mu^i_d(\mathbf{u}) &= \relp{i}{t}\\
    k_u &= 10\log(49)\\
    k_d &= \frac{\log(49)}{\sqrt{h^2 + l^2 + w^2}}
    \label{eq:ptransmissionInit}
  \end{align}
  is the distance of the centre of the TP from the camera.

  % $\mu^i_d(\mathbf{u}) = \min_{\lambda} d^2_i(\lambda K^{-1}\mathbf{u})$.
  % $\mu^i_d(\mathbf{u})$ is the closest point to the unit contour of ellipsoid.
  % If there are multiple such points, the point closest to the camera is taken as
  % $\mu^i_d(\mathbf{u})$

  The association probability becomes

  \begin{multline}
    \assocP = 
      \sech^4\left(\frac{k}{2}\dishort\right)
      (\max \{ 0, \nabla k\dishort^\top\ray \})^2\\
    \prod_i \Lu
      \Llambda \\
      \label{eq:assocCoeffEval}
  \end{multline}

  So the energy becomes

  \begin{multline}
    \label{eq:integrand}
    \Energy{track}(.) = 
      \sum_{i = 1}^N
      \sum_{j = 1}^{M}
      \int_1^{\infty}
      \assocP
      \Ereproj(\lambda)
      d\lambda
  \end{multline}
  where $x = \lambda \ray$ and $\Ereproj(\lambda) = \|\trackpj{t} -
  f^i_{reproj}(\trackpj{t-1}, \lambda)\|^2$ is reprojection error which is a
  quadratic in $\lambda$

  The integral in the above expression is computed numerically.
\end{comment}

\subsection{Object detection energy with occlusion} 

Object detection is usually followed by non-maximal suppression that results in
discarding similar bounding boxes. When we are jointly optimizing detections
with other cues, it is not usually desirable to go with a single bounding box.
Hence, we keep all the bounding box detections by approximating them with
multi-modal sum of Gaussian like logistic functions. We fit the parametric function of the form 
%
\begin{align}
  S(\bb{i}) = \sum_k A_k \exp(-(\bb{i}-\mu^{(d)}_k)^\top \Sigma^{(d)-1}_k
  (\bb{i}-\mu^{(d)}_k))
\end{align}
%
to detection scores, by non-linear error minimization with initialization from
non-maximal suppressed outputs. Here $\mu^{(d)}_j$ is one of the $k$ modes as a
4D vector representing a single bounding box as $[\minx, \miny, \maxx,
\maxy]^\top$. The optimization is constrained with symmetry and positive
definiteness of $\Sigma^{(d)-1}_k$, $\maxx \ge \minx$ and $\maxy \ge \miny$.

\subsubsection{Detection scores with occlusion reasoning} 
With our model of $\Ptrans$ described in Section \ref{sec:TPmodel}, we can
compute the probability of a point $\mathbf{u}$ on image be occluded assuming
the point is on TP $i$ with mean depth $\mu^{(i)}_d$ as
\begin{align}
  O_{i}(\mathbf{u}, \mu^{(i)}_d) = 1 - \Ptransmud \enspace .
\end{align}

If we a portion of our proposed detection bounding box is known to be occluded,
then we would like to decrease the confidence in the detection score about the
localization of that end of the object. Assuming that the occlusion is often on
the boundary of detection bounding boxes, we want to decrease our confidence on
the mean detection boundaries around the occluded boundaries. 
One of the simplest ways will be to scale the appropriate diagonal element of
$\Sigma_j$ by an appropriate scaling factor proportional to occlusion. But this
does not model appropriate how does occlusion affects the non diagonal terms.
Hence, we choose a covariance addition model where we compute a occlusion
covariance matrix, that provides a measure of occlusion in each direction.


To re-model our detection scores scaled by continuous occlusion we sample
$O_{i}(\mathbf{u}, \mu^{(i)}_d)$ at the hypothesized detection boundaries from
GMM $S(.)$ and we augment the detection boundary covariance matrix by
$\mathcal{P}_{j} = \rho_{j}\rho_{j}^\top$ where $\rho_{j} = O_{j}(\mathbf{u},
\mu^{(i)}_d)$. The new covariance matrix in detection score is given by 
%
\begin{align}
  \Sigma'^{(d)}_j = \mathcal{P}_{j} + \Sigma^{(d)}_j
\end{align}
%
The detection scores GMM with occlusion is given by replacing the covariance
matrix
%
\begin{multline}
  S'(\bb{i}) =\\
  \sum_j A_j \exp(-(\bb{i}-\mu^{(d)}_j)^\top \Sigma'^{(d)-1}_j
  (\bb{i}-\mu^{(d)}_j))
\end{multline}

The energy of detection scores is simply take to be the inverse of the detection score.
\begin{align}
  \Energy{detect}(\{ \relp{i}{t} \}_i, \{ \relp{i}{t-1} \}_i, \{\dimsn{i}\}_i ) = \frac{1}{S'(\bb{i})}
\end{align}

\subsection{Lane energy}
\label{sec:laneEnergy}
 The lanes are modeled as splines. Here we assume that the confidence in lane
 detection is decreases as the distance from the lane center increases.  The
 energy is given by the dot product between car orientation and tangent to the
 lane at that point.

\begin{multline}
  \label{eq:laneOrientationEnergy}
  \Energy{lane} = \\
  \sum_{m \in M_{\text{close}}}
  (1 - \ori{i}{t} \cdot \text{TAN}(L_{m}(k), \pos{i}{t}) )
\LaneUncertainty{\pos{i}{t}}
\end{multline}
where $M_{\text{close}} = \{m : \text{DIST}(L_{m}(k), \pos{i}{t}) < 50\} $ is
the set of nearby lanes and 
\begin{multline}
\LaneUncertainty{\pos{i}{t}} = \\
  \frac{1}{1 + exp(-q(w_{\text{road}} - \text{DIST}(L_{m}(k), \pos{i}{t})))}
\end{multline}
for some constant $w_{\text{road}}$ that represents the width of the road.

\subsection{Transition probability}
Dynamics constraints should not only enforce smooth trajectories, but also the
holonomic constraints.  The following energy adds a penalty if the change in
position is not in direction of previous orientation.

\begin{align}
  \label{eq:totalPosTransitionEnergy}
  \Energy{dyn-hol} = 1 - \ori{i}{t-1} \cdot (\pos{i}{t} - \pos{i}{t-1})
\end{align}

The following energy adds a penalty for change in position and orientation
but a penalty for change in velocity is much better approximation. However, in
a Markovian setting that would mean extending the state space of the car to
include velocity.

\begin{align}
  \Energy{dyn-ori} &= \|\ori{i}{t} - \ori{i}{t-1}\|^2\\
  \Energy{dyn-vel} &= \|(\pos{i}{t} - 2\pos{i}{t-1}) + \pos{i}{t-2}\|^2
\end{align}

As a result the dynamics are modeled by weighted combination of holonomic
constraint and smoothness constraints.

\begin{align}
  \WEnergy{dyn} &= \WEnergy{dyn-hol} + \WEnergy{dyn-ori} + \WEnergy{dyn-vel}
\end{align}

 
\subsection{Collision energy}

Bhattacharya coefficient $\int_a^b\sqrt{p(x)q(x)}dx$ is a measure of similarity
of two distributions $p(x)$ and $q(x)$. If we represent TPs as gaussians in
Birds eye view (BEV), then similarity is a measure of collision. Exactly
overlapping distribution results in coefficient as 1. 
%The analytical form of
%Bhattacharya coefficient has been taken from
%\url{http://like.silk.to/studymemo/PropertiesOfMultivariateGaussianFunction.pdf}

\begin{multline}
  \label{eq:collisionEnergyHellingerDistance}
  \EnergyCol =\\ -\log\left(
  A_{ij}
  e^{-\frac{1}{8}
    \left(\pos{i}{t} - \pos{j}{t}\right)^\top
    P^{-1}
    \left(\pos{i}{t} - \pos{j}{t}\right)
    }
    \right)
\end{multline}
where 
\begin{align}
  A_{ij} &= \frac{|\Sigma_i|^\frac{1}{4}|\Sigma_j|^\frac{1}{4}}
  {|P|^\frac{1}{2}}\\
  P &= \frac{1}{2}\Sigma_i + \frac{1}{2}\Sigma_j\\
\Sigma_i^{-1} &= R^\top_{\ori{i}{t}} \begin{bmatrix} 2/l_i & 0 \\ 0 & 2/w_i \end{bmatrix}
R_{\ori{i}{t}}
\end{align}


\subsection{Size Prior}

Prior can include among many other things the size prior on the car.

\begin{align}
  \label{eq:totalSizeEnergy}
  \Energy{size} &= (\dimsn{i} - \expDimsn)^\top\Sigma_{\expDimsn}^{-1}(\dimsn{i} -
  \expDimsn)
\end{align}

where $\expDimsn$ is the mean TP dimensions and
$\Sigma_{\expDimsn}$ is the correspondence covariance matrix.
