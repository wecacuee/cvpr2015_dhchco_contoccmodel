\section{Related Work}
\label{sec:related}


\paragraph{Occlusion handling}
Several works in object detection consider occlusion by training a detector on visible parts of the object \cite{Gao_etal_2011,Wu_Nevatia_2007}. Occlusion reasoning based on 2D image silhouettes is used to improve detection performance in \cite{Hsiao_Herbert_2012}.  In recent years, object detectors have also considered occlusion reasoning using 3D cues, often learned from a dataset of CAD models \cite{Pepik_etal_2012,Pepik_etal_2013,Xiang_Savarese_2013}. By necessity, such frameworks are often a discrete representation of occlusion behavior, for example, in the form of a collection of occlusion masks derived from object configurations discretized over viewpoint. In constrast to these works, our occlusion modeling is also fully 3D, but allows for a continuous representation. Further, to derive 3D information, we do not use CAD models, rather we derive a probabilistic formulation based on physical insights.


Occlusion handling in tracking \cite{Kwak_etal_2012,Wu_Nevatia_2007,Milan_etal_2014}.


\paragraph{3D localization and scene understanding}
General \cite{Geiger2011, Geiger_etal_2014}.
Occlusion handling in scene understanding works \cite{Wojek_etal_2013,Zia_etal_2013,Zia_etal_2014}.


\paragraph{Motion segmentation and multibody SFM}
\cite{Tron_Vidal_2007}
\cite{Rao_etal_2010}
\cite{Brox_Malik_2010}


\cite{Schindler2006}
\cite{Ozden_etal_2010}
\cite{Namdev2012}
\cite{Kundu_etal_2011}



\paragraph{Motion segmentation}






TODO:
\vfill
\pagebreak
Related work continued
\vspace{15cm}

