\section{Modeling traffic participants}
\label{sec:TPmodel}
Following standard model of occupancy in 3D reconstruction community, we model
the TP's as high occupancy regions in space. We model the
uncertainty in occupancy equivalent of transparency i.e. the regions that are
more certain to be occupied are viewed as relatively more opaque then other
regions. Based on this intuition we model TP's translucent
ellipsoids whose opacity is maximum at the center and drops off away from the
center. More specifically we model, the occupancy of a TP as a
logistic function, 

\begin{align}
   \occf = L(\mathbf{x}; \pos{i}{t}, \Sigma_i)
\end{align}
where $L(.)$ is the logistic function defined by
\begin{align}
  L(\mathbf{x}; \pos{i}{t}, \Sigma_i) = \frac{1}{
    1 + e^{-k(1 - d(\mathbf{x},\pos{i}{t}))}
    }
\end{align}
where $d(\mathbf{x},\pos{i}{t}) =
(\mathbf{x}-\pos{i}{t})^\top\Sigma_i(\mathbf{x}-\pos{i}{t})$ and $k =
10\ln{49}$. $k$ is chosen such that $L(.) = 0.98$ when $d(.) =
0.9$. $\Sigma_i$ determines the spread of the ellipsoid and depends on the
dimensions of the TP. Please refer to supplementary material 
for computation of $\Sigma_i$ from TP's dimensions.

We model the probability of a point $\mathbf{x}_j$ on a object $i$ getting successfully
observed in a camera image at point $\trackpj{t}$ is dependent up two factors,
(1) reflection and (2) transmission through intermediate space. The reflection
part ensures that there is an object to reflect a point at a certain region in
the 3D space while transmission part models occlusion.
\begin{align}
  P^{(ij)}_{\text{observation}} = P^{(ij)}_{\text{reflection}}P^{(j)}_{\text{transmission}}
\end{align}

%%%%%%%%%%%%%%%%%%%%%%%%%%%%%%%%%%%%%%%%%%%%%%%%%%%%
\subsection{Reflection probability}
For Lambertian reflection we replace the surface normal with the
gradient of occupancy.
%
\begin{align}
  \Prefl = (\max \{0, \nabla \occf^\top
  \hat{\mathbf{r}_j}\})^2
\end{align}
%
where $\ray =
\frac{K^{-1}\trackpj{t}}{\|K^{-1}\trackpj{t}\|}$ is unit vector in the
direction of ray. The gradient in the direction opposite to ray yields -ve
probability which needs to be clipped off. Squaring the function keep it
smooth near zero.

%%%%%%%%%%%%%%%%%%%%%%%%%%%%%%%%%%%%%%%%%%%%%%%%%%%%
\subsection{Transmission probability}
A model for transmission of light through a material of thickness $x$,
density $\rho$ and opacity $k_o$ is given by Beer-Lambert law 
%
\begin{align}
  I(x) = I_0e^{-k_o\rho x}
\end{align}
%

Since both opacity and density are represented by the occupancy function
$\occftot = \sum_i \occf$, and also the domain of our $\occftot$ is $[0, 1]$ instead of $[0,
\infty]$ as in case of $k_o$; we replace $e^{-k_o\rho}$ by the transparency
function $1 - \occftot$. So the transmission probability over a small distance
$d\lambda$ is given by
%
\begin{align}
  P_{\text{transmission}}(\lambda + d\lambda) =
  P_{\text{transmission}}(\lambda) (1-\occftot)^{d\lambda}
\end{align}
%

For a given 3D point $\mathbf{x}_j = \lambda \ray$, the probability that the
point $\trackpj{t}$ is reflected from a distance $\lambda$ is given by

\begin{align}
  %P^{(j)}_{\text{observation}}(\lambda) &= P_{\text{reflection}}
  \Ptrans &=
  \prod_{0}^{\lambda} (1 - \occft{\lambda \ray})^{d\lambda} %\\
  %= \max \{ 0, (\nabla f_{occ}&(\lambda \ray)^\top \ray) \}
  %\prod_{0}^{\lambda} (1 - f_{occ}(\lambda \ray))^{d\lambda}
  \label{eq:ptrans-integral}
\end{align}
where $\prod_{0}^{\lambda}$ represents the \emph{product integral} from $0$ to
$\lambda$. 

In practice, the integral for transmission probability
\eqref{eq:ptrans-integral} is difficult to compute even numerically. So we
choose a product of sigmoid function that approximates the behaviour of
transmission probability,
%
\begin{align}
\label{eq:evalCumulativePtrans}
  \Ptrans &= \prod_i L_u(\trackp{t}; \muiu,\Sigmaiu)L_{\lambda}(\lambda; \mu^i_d)
\end{align}
%
where $L_u(.)$ is sigmoid in image domain with $\mu^i_u$ and $\Sigma^i_u$
representing the elliptical projection of $i^{th}$ TP.
$L_{\lambda}(.)$ is sigmoid in the depth domain with $\mu^i_d$ as the mean
depth of the $i^{th}$ TP.
%
\begin{align}
  L_u(\mathbf{u}; \muiu,\Sigmaiu) &= \frac{1}{
    1 + e^{-k_u(1 - (\mathbf{u} - \muiu)^\top\Sigmaiu(\mathbf{u} -
    \muiu))}
  }
  \\
  L_{\lambda}(\lambda; \mu^i_d) &= \frac{1}{
    1 + e^{-k_d(\lambda - \mu^i_d)}
}
\end{align}
%

\begin{comment}%% Comment
  A product integral is a simple integral in log domain
  \begin{align}
    \prod_{0}^{\lambda} (1 - f_{occ}(\lambda \ray))^{d\lambda} =
    e^{\int_{1}^{\lambda} \ln{(1 - f_{occ}(\lambda \ray))}{d\lambda}}
  \end{align}
\end{comment}%% Comment

