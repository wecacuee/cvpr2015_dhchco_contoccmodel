\begin{abstract}

We present a physically interpretable, continuous 3D model for handling occlusions with applications to road scene understanding. We probabilistically assign each point in space to an object with a theoretical modeling of the reflection and transmission probabilities for the corresponding camera ray. Our modeling is unified in handling occlusions across a variety of scenarios, such as associating structure from motion point tracks with potentially occluded objects or modeling object detection scores in applications such as 3D localization. For point track association, our model uniformly handles static and dynamic objects, which is an advantage over motion segmentation approaches traditionally used in multibody SFM. Detailed experiments on the KITTI dataset show the superiority of the proposed method over both state-of-the-art motion segmentation and a baseline that heuristically uses detection bounding boxes for resolving occlusions. We also demonstrate how our continuous occlusion model may be applied to the task of 3D localization in road scenes.

%This paper studies the problem of segmenting tracked feature points on multiple objects in a video sequence. We consider scenes that contain both dynamic and static objects, which usually appear in the context of autonomous driving (e.g., moving and parked cars), and occlusions among objects. We propose a novel continuous model for modeling objects so that point tracks can be probabilistically and accurately assigned to the corresponding object while accounting for object occlusions. Detailed experiments on the KITTI dataset show the superiority of the proposed method over the baseline method which simply uses detection bounding boxes for resolving occlusions and the state-of-the-art motion segmentation methods which cannot cope well with static objects. As a notable application, we show how our continuous occlusion model benefits the task of 3D localization in autonomous driving.  
\end{abstract}
