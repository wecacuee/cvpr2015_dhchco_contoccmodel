\section{Conclusions and Future Work}
\label{sec:conclusions}

We have presented a theoretically novel continuous model for occlusion reasoning in 3D. A key advantage is its physical inspiration that lends flexibility towards occlusion reasoning for varied elements of scene understanding, such as point tracks, object detection bounding boxes and detection scores. We demonstrate unified modeling for different applications such as point tracks associations and 3D localization. Our occlusion model can uniformly handle static and dynamic objects, which is an advantage over motion segmentation methods for point-object association. A challenge is that inference for 3D localization is currently slow, requiring a few minutes per window of frames, which prevents exhaustive cross-validation for tuning of weights. Our future work will explore speeding up the inference, for example, by approximating the graph with a tree using the Chow-Liu method \cite{chow1968approximating}, allowing belief propagation for fast inference.



%Our experiments show that our association probability produces more accurate point to object associations when compared to bounding box based association and state of art motion segmentation. Also the localization experiment results show improvement in localization accuracy by modeling occlusions. However, the average translation error of 5 meters is very high. Since the results are averaged over the entire KITTI dataset, there are various possible sources of error, including outliers in point tracks, outliers in detections results.  One of the reasons of poor performance is that we set the parameters $\lambda_{\text{bbox,track,lane,dyn,size}}$ empirically which may be suboptimal. However, because of slow inference algorithms, which take about a day to run on 300 cores for the entire KITTI dataset, learning become infeasible.

%We have several ideas for speeding up inference and thereby allowing learning of weights to become feasible. One of them is to approximate the graph into a tree using Chow-Liu's \cite{chow1968approximating} algorithm and then using Belief Propagation which is linear in the number of edges of the tree.
