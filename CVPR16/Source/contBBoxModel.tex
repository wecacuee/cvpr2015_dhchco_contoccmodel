\vspace{-0.3cm}
\paragraph{Continuous bounding box energy with occlusion}
Object detection is usually followed by non-maximal suppression that results in discarding similar bounding boxes. When we are jointly optimizing detections with other cues, it is not usually desirable to go with a single bounding box. To retain the entire detection output while maintaining the continuous form of our energies, we approximate the distribution of detection scores with a multi-modal sum of Gaussians-like logistic functions. In particular, let 2D bounding box $\bb{i}$ be parameterized as a 4D vector $[\minx, \miny, \maxx, \maxy]^\top$. We fit a parametric function to the detection scores, of the form
\begin{equation}
S(\bb{i}) = \sum_k A_k \exp \left( -\bepsilon_k^{i}(t)^\top \Lambda_k^{-1} \bepsilon_k^{i}(t) \right)
\end{equation}
where $A_k$ is an amplitude and $\bepsilon_k^{i}(t) = \bb{i}-\mu^{k}_d$, with $\mu^{k}_d$ the mean and $\Lambda_k$ the covariance.
%% %
%% \begin{multline}
%%   S(\bb{i}) = \\
%%   \sum_k A_k \exp(-(\bb{i}-\mu^{(d)}_k)^\top \Sigma^{(d)-1}_k
%%   (\bb{i}-\mu^{(d)}_k))
%% \end{multline}
%% %
%% to detection scores, by 
We use a non-linear solver to minimize the above, with initialization from non-maximal suppressed outputs. The optimization is constrained by the symmetry and positive definiteness of $\Lambda_k^{-1}$, $\maxx \ge \minx$ and $\maxy \ge \miny$.
%Here $\mu^{(d)}_j$ is one of the $k$ modes as a 4D vector representing a single bounding box as $[\minx, \miny, \maxx, \maxy]^\top$. The optimization is constrained with symmetry and positive definiteness of $\Sigma^{(d)-1}_k$, $\maxx \ge \minx$ and $\maxy \ge \miny$.

\paragraph{Detection scores with occlusion reasoning} 
\def\u{\mathbf{u}}
With our model of $\Ptrans$ described in Section \ref{sec:ptransmission}, we can
compute the probability of a point $\u$ in the image to be occluded assuming
the point is on object $O_i$ with mean depth $\meandepth{i}$ as
\begin{align}
  \Theta_{i}(\u, \meandepth{i}) = 1 - \Ptransmission(\meandepth{i}, \u) \enspace .
\end{align}

If a portion of our proposed detection bounding box is known to be
occluded, then we would like to decrease the confidence in the detection score
about the localization of that end of the object. Assuming that the occlusion
is often on the boundary of detection bounding boxes, we want to decrease our
confidence on the mean detection boundaries around the occluded boundaries.
To re-model our detection scores scaled by continuous occlusion we sample
$\Theta_{i}(\mathbf{u}, \meandepth{i})$ at the hypothesized detection boundaries from
Gaussian mixture model (GMM) $S(.)$ and we augment the detection boundary covariance matrix by
$\mathcal{P}_{i} = \rho_{i}\rho_{i}^\top$ where $\rho_{i} = \Theta_{i}(\mathbf{u},
\meandepth{i})$. The new covariance matrix in detection score is given by 
  $\Lambda'_k = \mathcal{P}_{i} + \Lambda_k$ for all $k$.
The detection scores GMM with occlusion is given by replacing the covariance
matrix
%
\begin{multline}
S'(\bb{i}) = \sum_k A_k \exp \left( -\bepsilon_k^{i}(t)^\top \Lambda_k^{'-1} \bepsilon_k^{i}(t) \right)
\end{multline}

The energy of detection scores is simply take to be the inverse of the detection score.
\begin{align}
  \Energy{detect}(\{ \relp{i}{t} \}_i, \{ \relp{i}{t-1} \}_i, \{\dimsn{i}\}_i ) = \frac{1}{S'(\bb{i})}
\end{align}

%For object detections we use object detector by \cite{Felzenszwalb_etal_2010} which is detector by parts model and we use eight parts to train the car model on half of the KITTI dataset \cite{geiger2013vision}. The trained modeled is used to get detections for the other half of the dataset and vice versa. 
