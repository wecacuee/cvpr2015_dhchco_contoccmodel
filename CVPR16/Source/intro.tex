\section{Introduction}
The problem of segmenting points to objects often comes up in different
computer vision problems especially in multi-body 
reconstruction and localization. Given correspondence points on
consecutive frames, the problem becomes one of finding out which points belong
to which object, following which one can use the consensus of segmented points
to reconstruct or localize the object more accurately. Without additional 
information, this is a very difficult problem as attempted by semantic segmentation 
and motion segmentation methods. However, in the problems like reconstruction and
localization one can use the hypothesis of the algorithm as the feedback.
% What is the problem we are trying to solve, and why is it even important?

% What did we do
We propose a novel model for representation of objects in the scene so that point tracks
can be probabilistically associated with objects while accounting for object occlusions.
% Why is it novel
Although our proposed model is inspired by Milan et
al.\cite{Milan_etal_2014}, it is more detailed and more principled than their
work. Milan et al. used soft ellipses and sigmoids to model pedestrians and
their occlusions. Our reasoning is instead in 3D. We use ellipsoids to model objects
and reason about their occupancies of the space to model their occlusions.
% Ok you proposed a model is it of any use
% How does it compare to state of art
We show that our continuous occlusion model achieves significantly lower point-to-object association errors compared to the baseline method using detection bounding boxes and other state-of-the-art methods for motion segmentation~\cite{Brox_Malik_2010, Rao_etal_2010}. As a consequence, our model benefits the task of 3D localization of traffic participants in autonomous driving.
