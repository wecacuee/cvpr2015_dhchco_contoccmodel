\section{Object detection energy}

We model object detection energy as the norm of difference in the coordinates
of the projected bounding box and detected bounding box. The detected bounding
box is represented by the extrema along X and Y dimensions $\bb{i} = [\xmin,
\ymin, \xmax, \ymax]^\top$. If $\dimsn{i} = [l, w, h]^\top$ represent the 3D
dimensions of the traffic participant along the X, Y and Z dimensions then we 
can compute the projection of the bounding box in terms the dimensions $\dimsn{i}$
and object pose $\Omega_i(t)$. We model occlusion such that the confidence in
accuracy of the particular bounding box edge is reduced proportional to its
occlusion. We measure the extent of occlusion by the fraction of the area of 
the triangle corresponding to the occluded side and call it visibility fraction.
The visibility fraction stacked as a diagonal matrix gives us visibility matrix
$\mathcal{V}$ which can be used to scale the difference between projected bounding box
$\projectionOf{\dimsn{i}}$ and the detected bounding box $\bbt{i}{t}$. More precisely, for
the situation described in Fig~\ref{fig:bboxoccfigure}.
%
\begin{align}
  \mathcal{V} &= \text{diag}([v_{AB}, v_{BC}, v_{CD}, v_{DA}])\\
  \EnergyBBox &= \left\| \mathcal{V} \left(\projectionOf{\dimsn{i}}-\bbt{i}{t}\right) \right\|^2 
\end{align}
%
%
\begin{figure}
  \centering
  ../presentation/bboxoccfigure.tex
  \caption{We model the occlusion of bounding by taking the corresponding
  triangle as a proxy for edge. The visibility fraction for the AD, $v_{AD}$ is given
by the fraction of the area of triangle $\bigtriangleup AED$ 
that is occluded.}
  \label{fig:bboxoccfigure}
\end{figure}

% \begin{align}
%   X^{(i)}_{\text{cuboid}} &= \dimsn{i}{}^\top C_u \\
%   \text{where} &\\
%   C_u &= 
%   \begin{bmatrix}
%   0.5 & 0.5 & -0.5 & -0.5 & 0.5 & 0.5 & -0.5 & -0.5 \\
%     0.5 & -0.5 & -0.5 & 0.5 & 0.5 & -0.5 & -0.5 & 0.5 \\
%     0 & 0 & 0 & 0 & 1  & 1 & 1 & 1
%   \end{bmatrix} \enspace .
% \end{align}
% These cuboid points can be projected to the camera image plane to get the projected bounding box,
% \newcommand{\ucub}{U^{(i)}(t)}
% \newcommand{\ucubx}{U_{(1,:)}^{(i)}(t)}
% \newcommand{\ucuby}{U_{(2,:)}^{(i)}(t)}
% \newcommand{\estbb}{\hat{\mathbf{d}}^{(i)}(t)}
% \begin{align}
%   \ucub &= \projectionOf{\dimsn{i}{}^\top C_u}\\
%   \estbb &= [\min \ucubx, \min \ucuby, \max \ucubx, \max \ucuby]^\top \enspace .
% \end{align}
% We take $\estbb$ as the projected bounding box. The object detection energy is
% taken to be the euclidean distance between projected and detected bounding box.
% \begin{align}
%   \EnergyBBox &= \| \estbb - \bb{i}\|^2_2
% \end{align}
% With slight abuse of notation for projection operator we write this energy as
% \begin{align}
%   \EnergyBBox &= \left\| \projectionOf{\dimsn{i}} - \bb{i}\right\|^2_2 \enspace.
% \end{align}
